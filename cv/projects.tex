%-------------------------------------------------------------------------------
%	SECTION TITLE
%-------------------------------------------------------------------------------
\cvsection{University Projects}


%-------------------------------------------------------------------------------
%	CONTENT
%-------------------------------------------------------------------------------
\begin{cventries}

%---------------------------------------------------------
  \cventry
    {Microelectronic system course} % Organisation
    {Design of a DLX processor} % Project
    {Politecnico di Torino, IT} % Location
    {2021} % Date(s)
    {
      \begin{cvitems} % Description(s) of project
        \item{RTL design, synthesis and physical design of a 32-bit RISC core using VHDL and EDA tools (ModelSim, Synopsys and Innovus)}
        \item {Load-Store architecture with 4 pipeline stages.}
        \item {Integer subset of the DLX ISA.}
        \item {Customized ALU implemented using optimized arithmetic blocks of the Pentium 4 and UltraSPARC T2 processors.}
        \item {Basic Hazard Detection features.}
        \item {Software automation toolchain for testing the design at RTL-level with custom assembly programs.}
        \item {Synthesis scripts using Synopsys commands.}
        \item {\textbf{Technical Skills:} ASIC design, EDA tools.}
        \item {\textbf{Soft Skills:} Time Management, Teamwork, Presentation skills, Report writing.}
      \end{cvitems}
    }


%---------------------------------------------------------
  \cventry
    {Testing and Fault Tolerance course} % Organisation
    {Design of a LBIST circuit for testing the RI5CY processor} % Project
    {Politecnico di Torino, IT} % Location
    {2022} % Date(s)
    {
      \begin{cvitems} % Description(s) of project
        \item{Design of a synthesizable LBIST wrapper for a RISC-V processor with the purpose of covering a given percentage of stuck-at faults}.
        \item{Adoption of Test-per-Scan methodology.}
        \item {Evaluation of the LBIST impact in terms of area and time required to run a complete test.}
        \item {\textbf{Technical Skills:} testing of big sequential circuits, ATPG, Test-per-Scan technique, TestMAX, SpyGlass.}
        \item {\textbf{Soft Skills:} Presentation skills, Teamwork.}
      \end{cvitems}
    }

%---------------------------------------------------------
  \cventry
    {GPU Programming course} % Organisation
    {Acceleration of a CNN in CUDA for the GPU Nvidia Tegra-X1} % Project
    {Politecnico di Torino, IT} % Location
    {2022} % Date(s)
    {
      \begin{cvitems} % Description(s) of project
        \item {Critical evaluation of the CNN's forward propagation accelerated performances with respect to the CPU version.}
        \item {Methodological approach for CUDA-C programming.}
        \item {Basic knowledge of Convolutional Neural Networks.}
        \item {Knowledge of nvprof command line profiler.}
        \item {GPU Nvidia Maxwell architecture}
        \item {\textbf{Technical Skills:} Markdown, Git, CUDA, Nvprof, Nvidia GPUs, LaTex.}
        \item {\textbf{Soft Skills:} Report writing, Critical Thinking, Presentation skills.}
      \end{cvitems}
    }

%---------------------------------------------------------
  \cventry
    {Final Project Work} % Organisation
    {MC2101: A RISC-V-based Microcontroller for Security Assessment and Training} % Project
    {Politecnico di Torino, IT} % Location
    {2022} % Date(s)
    {
      \begin{cvitems} % Description(s) of project
        \item {The purpose of my thesis was to design the entire architecture of a simple embedded system, compatible with the RISC-V ISA, that can be synthesized on FPGA and used on a development board.}
        \item{RTL design of the bus infrastructure, memory and peripherals}
        \item{Software design of system libraries and bootloaders}
        \item {\textbf{Technical Skills:} LaTex, Git, Microcontrollers architecture, Low-power, RISC-V, Quartus Prime, FPGA, C, ASM, VHDL}
        \item {\textbf{Soft Skills:} Report writing, Critical Thinking, Presentation skills.}
      \end{cvitems}
    }

%---------------------------------------------------------
\end{cventries}
